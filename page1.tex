\documentclass[a4,portrait]{template/a0poster}

\usepackage{multicol} % This is so we can have multiple columns of text side-by-side
\columnsep=100pt % This is the amount of white space between the columns in the poster
\columnseprule=3pt % This is the thickness of the black line between the columns in the poster

\usepackage[svgnames]{xcolor}
\usepackage{graphicx} % Required for including images
\usepackage{times}
\usepackage[font=small,labelfont=bf]{caption} % Required for specifying captions to tables and figures
\usepackage{amsfonts, amsmath, amsthm, amssymb} % For math fonts, symbols and environments

\begin{document}

\begin{minipage}[b]{\linewidth}
\VeryHuge \color{NavyBlue} \textbf{Can we date an artist's work from catalogue photographs?}
\color{Black}\\ % Title
\huge \textbf{Alexander D. Brown}, Gareth L. Roderick, Hannah M. Dee, Lorna M.
Hughes\\[0.5cm] % Author(s)
\Large \texttt{adb9@aber.ac.uk}\\
\end{minipage}
\vspace{1cm} % A bit of extra whitespace between the header and poster content

\begin{multicols}{2}

\Large
\section*{Abstract}
Computer vision has addressed many problems in art, but has not yet looked in
detail at the way artistic style can develop and evolve over the course of an
artist's career. In this paper we take a computational approach to modelling
stylistic change in the body of work amassed by Sir John “Kyffin” Williams, a
nationally renowned and prolific Welsh artist. Using images gathered from
catalogues and online sources, we use a leave-one-out methodology to classify
paintings by year; despite the variation in image source, size, and quality we
are able to obtain significant correlations between predicted year and actual
year, and we are able to guess the age of the painting within 15 years, for
around 70\% of our dataset. We also investigate the incorporation of expert
knowledge within this framework by considering a subset of paintings chosen as
exemplars by a scholar familiar with Williams' work.

\section*{Background}

\section*{Introduction}

\section*{The Image Dataset}
Our image dataset consisted of 325 paintings, including associated metadata for
each painting. This metadata includes title; year or year range, for those
works which the actual date is unknown but can be reasonably estimated by
curators; genre; original painting size; painting material and image size.

Because many of these images were collected from catalogue photographs there
was no control over the lighting conditions in which they were captured nor the
resolution of these images.

\section*{Methodology}

\section*{Colour Features}

\section*{Texture Features}

\section*{Year Classification Results}

\section*{Exemplars}

\section*{Conclusions}

\end{multicols}

\end{document}

